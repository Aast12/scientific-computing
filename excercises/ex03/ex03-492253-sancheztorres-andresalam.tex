\documentclass{article}
\usepackage[english]{babel}
\usepackage[utf8]{inputenc}
\usepackage[T1]{fontenc}
\usepackage[a4paper, margin=1in]{geometry}
\usepackage{relsize}
\usepackage{amsfonts}
\usepackage{amsthm}
\usepackage{amssymb}
\usepackage{mathtools}
\usepackage{titlesec}
\usepackage[shortlabels]{enumitem}


\title{Scientific Computing \\[0.2em]\smaller{}Assigment 3}
\author{Andres Alam Sanchez Torres}
\date{November 10, 2023}

\begin{document}
\maketitle

\section*{Problem 1}

\begin{align*}
  00100101_2 &= 1 \cdot 2^0 + 
               0 \cdot 2^1 + 
               1 \cdot 2^2 +
               0 \cdot 2^3 +
               0 \cdot 2^4 +
               1 \cdot 2^5 +
               0 \cdot 2^6 +
               0 \cdot 2^7 \\
             &= 1 + 
               4 +
               32\\
             &= 37 \\
  \text{Two's Complement:}&~11011010_2 + 1_2\\
                          &~11011011_2 \\ \\
  01110101_2 &= 1 \cdot 2^0 + 
             0 \cdot 2^1 + 
             1 \cdot 2^2 +
             0 \cdot 2^3 +
             1 \cdot 2^4 +
             1 \cdot 2^5 +
             1 \cdot 2^6 +
             0 \cdot 2^7 \\
           &= 1 + 
             4 +
             16 +
             32 +
             64\\
           &= 117 \\
  \text{Two's Complement:}&~10001010_2 + 1_2\\
           &~10001011_2 \\ \\
  \shortintertext{With the same procedure:}
  0101100010010110_2 &= 22678 \\
  \text{Two's Complement:}&~1010011101101010_2\\ \\
  0110100110100110_2 &= 27046 \\
  \text{Two's Complement:}&~1001011001011010_2\\
\end{align*}


% 2. Patterns

\section*{Problem 2}

\begin{enumerate}
  \item How many different bit patterns can be represented using 63 bits? \\
        $2^{63}$
  \item If we want to store any integer $x$ where $0 \leq x \leq 25$.
        What is the smallest number of bits we can use? \\
        $\lceil log_2(25) \rceil = 5$ \\
        To elaborate further, $25 = 11001_2$, so we need at least 5 digits to represent 25 in binary.
  \item If we represent colours as red-green-blue (RGB) triples, where we use 4 bits for red, 4 bits for
  green, and 4 bits for blue. How many different colours can we represent? \\
      As a whole, a color will be represented by 12 bits, which can represent $2^{12}$ colors.
  \item If we represent a decimal number (like 13579) by concatenating the binary representations of
  each decimal digit. How many bits would we need for a number which contains $n$ digits?

  In general,\\
  \begin{equation*}
   \sum_{i = 0}^{n}
      \begin{cases}
        d_i \in \{0, 1\}& 1\\
        \text{other} & \lceil log_2(d_i) \rceil \\
     \end{cases}
    \end{equation*}
  Where $d_i$ is the ith digit of the decimal number. If we use a fixed set for each digit
  we will need 4 bits per digit (since the maximum value for a digit, 9, requires such bits). Then, 
  we will need $4 \cdot n$ bits. \\

\end{enumerate}



\end{document}
